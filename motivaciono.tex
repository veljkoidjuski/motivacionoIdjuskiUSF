\documentclass[a4paper]{article}

\usepackage[T1]{fontenc}

\usepackage[margin=1in]{geometry}

\hyphenation{pred-sta-vlja-lo}

\begin{document}

\noindent Veljko Iđuški

\noindent Srednja tehnička škola Sombor

\noindent 25000 Sombor, Republika Srbija

\noindent email: \verb!veljkoidjuski@gmail.com!\\

\vspace{1em}

{\centering \textbf{Motivaciono pismo}

u sklopu prijave za naučnu konferenciju „Uzbudljivi svijet fizike“\\}

\vspace{2em}

{\centering Poštovani,\\}

\vspace{1em}

Čoveka pokreće spoznaja. On teži da razume i opiše čudesan svet u kom se našao, takoreći, igrom slučajnosti; kao razumno biće, on bezuslovno stremi tome da objasni samom sebi sve ono što zapazi u okruženju, stvari naizgled od strane neke više sile propisane, i, samim tim, nedokučive i neupitne. Ostvarenje ovog čovekovog nastojanja jeste nauka koju nazivamo fizikom.\\

%Ona je, danas, sveprisutna. Kao osnov sve tehnike, njeno %proučavanje neizostavni je sačinilac sveg ljudskog %napretka; nova saznanja

Nisam uvek umeo razumeti lepotu fizike; premda sam bio svestan da u dubini krije nešto dobro i krasno, kroz osnovnu školu ona nije u mojim očima bila mnogo više od naročite zbirke ne naročito smislenih matematičkih obrazaca, na koju sam gledao s nekom vrstom strahopoštovanja umesto s ljubavlju i oduševljenjem. Iako sam se sa zadacima snalazio sasvim dobro, pripremanje za takmičenje predstavljalo mi je preterano velik napor, usled suštinskog nerazumevanja materije. Međutim, s početkom svog srednjoškolstva upoznao sam istinsko lice fizike; najednom, shvatio sam suštinu ove nauke. Spoznao sam njenu sveprisutnost i univerzalnu primenljivost, njenu ulogu pokretača i neizostavnog sačinioca sveg napretka tehnike, njenu mogućnost da sistematizuje ovaj neobičan svet. Ove četiri stavke jesu zaključci do kojih sam došao u pokušaju da samom sebi kažem šta je, u stvari, to što je uslovilo moju današnju zaljubljenost u fiziku.\\

Mada bi se možda takav čin smatrao prigodnim, ne bi bilo pošteno niti iskreno od mene da posebno ističem svoju naklonost prema ijednoj oblasti fizike. Verujem da svaka oblast fizike ima neku svoju čar, nešto što bi me moglo zainteresovati i izazvati da je dublje istražim. Sticanje novih znanja u oblastima koje me prethodno nikad nisu posebno privukle ili o stvarima za koje nisam mogao ni da zamislim da postoje ili da su moguće --- to me uvek učini srećnim, zadovoljnim i uzbuđenim. Doduše, nije naodmet napomenuti da najbolje poznajem elektrokinetiku, budući da ona čini osnov mog srednjoškolskog obrazovanja, a pored nje smatram da izuzetno dobro poznajem i elektrostatiku, magnetizam, kinetičku teoriju gasova i termodinamiku. U skorije vreme, moju pažnju posebno privlači teorija haosa u fizici, usled njene prividne paradoksalnosti.\\

U slobodno vreme, nisam mnogo više od običnog srednjoškolca. Volim da se bavim računarima, pogotovo starijim, imam i omanju kolekciju komponenti koju težim da proširim. Počeo sam se prošlog leta povremeno baviti i elektronikom kao hobijem, a još od četvrtog razreda osnovne škole uživam i u robotici. Izuzetan sam ljubitelj lepe književnosti, a kao omiljenu knjigu uvek navodim 1984 --- ta knjiga mi se posebno dopada jer nosi jednu svevremenu pouku. Uživam u muzici; nemam strogo određene žanrove koje volim, već slušam štagod mi se dopadne. Zbog ljubavi prema muzici, u osnovnoj školi poželeo sam da naučim svirati klavir; završio sam nižu muzičku školu i, iskreno, žao mi je što nisam upisao i višu. U sviranju klavira nalazim izuzetno zadovoljstvo i uživanje. Pored svega toga, treniram plivanje, a ranije sam trenirao i rukomet.\\



Kako bih zadovoljio potrebe svog istraživačkog duha, redovno učestvujem na naučnim kampovima i seminarima. Rado pričam o svojim dosadašnjim iskustvima, te sam pomislio kako ne bi bilo zgoreg ovde kratko nabrojati i opisati moja iskustva. Dosad sam bio polaznik šest seminara u Istraživačkoj stanici Petnica, od kojih je prvi bio oktobra 2021. Trenutno sam stariji polaznik seminara računarstva, a sem ovog programa bio sam polaznik Letnje naučne škole i kombinovanog seminara matematičko-tehničkih nauka. Od ove godine sam polaznik kampova udruženja Primenjena fizika i elektronika u Tršiću, a na istom mestu sam pohađao i seminar lingvistike 2023, kao i dva seminara srpskog jezika za učenike prvoplasirane na republičkim takmičenjima iz tog predmeta (2022. i 2023.). U januaru ove godine učestvovao sam na kampu fizike u Sokobanji u organizaciji Udruženja fizičara Omega iz Niša. Uprkos činjenici da su pobrojani seminari poprilično raznoliki u pogledu oblasti kojima su se bavili, svi ovi seminari za mene su bili nezaboravna i neponovljiva iskustva, na svakom od njih mnogo sam toga novog saznao, stekao mnogo novih prijatelja, otišao s mnogo novih iskustava. To, za mene, jeste cela svrha učešća na svim ovim seminarima. Iz tog razloga se zaista nadam da ću ubrzo biti u prilici ovom podužem spisku pridodati i svoje učešće na vašem seminaru.\\

U Somboru, 30. juna 2024.
\end{document}